\section{Background}
\label{sec:Background}

\subsection{Digital Forensics and the use of Machine Learning}

A perpetrator always leaves traces of evidence of their involvement at the crime scene, as described by Dr. Edmond Locard in his exchange principles, which are used in forensic science [DF 3].
\newline
The landscape of forensic science has changed due to the rise of electronic devices, which play an increasingly important role in our daily lives and are often connected to the internet and accessible from anywhere.
The field has expanded since the early 2000s with digital forensics (DF), which specializes more in the growing number of cybercrimes.
However, even crimes that are not classified as cybercrime are becoming increasingly digital in most modern crime scenes.
According to the EU, digital evidence is involved in 85\% of criminal investigations.
This evidence consists of data generated in our daily lives through the use of digital devices, leaving behind a digital footprint.
The footprint consists of data generated by wearable devices, emails, cloud service providers, online payments, and other sources [DF 1].
\newline
The field of digital forensics can be divided into seven identifiable sub-areas, namely blockchain, networks, mobile, cloud, IoT, file systems \& data storage, and multimedia.
This project is limited to the sub-area of multimedia, which specializes in image forgery [DF 1].
\newline
A major challenge in this field is dealing with large, complex data sets and classifying them.
This problem can be addressed by machine learning (ML), whose techniques have expanded and improved in recent years.
ML techniques search through the data and look for anomalies and patterns in the investigation process.
The largest area for ML in DF is image forensics, accounting for 62.7\%.
In the field of image forensics, convolutional neural networks (CNNs) are typically used to recognize such complex patterns in the data, 
which is why this approach was pursued in the project. [DF 3]

\subsection{CNNs for face fake detection}

\subsubsection{Fundamentals}

In recent years, there has been intensive research into CNNs in the areas of image classification, facial recognition, and facial expressions, resulting in significant improvements.
However, with the emergence of increasingly sophisticated deepfakes, it is becoming more and more difficult to distinguish between real and fake faces [CNN 1].
During the training process, CNNs learn complex patterns and can provide information about whether an input image is fake or real.
Among CNN architectures, EfficientNet is one of the most modern models, as it is faster, has fewer parameters and has more capabilities for extracting features than other CNN models.
These parameters are also called weights and can be learned by the model during the training process.
During the training process, the model is shown many input images of positive (real faces) and negative (fake faces)~\cite{wang2019development}.
The model independently learns complex patterns in the form of weights over several iterations (epochs) in order to distinguish positive from negative examples in the case of a binary classifier, or to assign probabilities to classes in the case of a multiple classifier system.~\cite{lorena2008review}\cite{hadjadji2014multiple}.
What is special about CNNs are the convolutional layers, which reduce the resolution of the images and extract the spatial local features through weighted convolutions.
Such local features can be low-level features such as edges, end points and corners in the first convolutional layers and complex high-level features in the last layers.
In the final layers, the high-level features (3D vector) are combined into a fully connected layer (1D) to make a classification decision.
In this project, the output consists of two output nodes with probabilities indicating whether the input image is a fake or real face~\cite{wang2019development}.

\subsubsection{Transfer Learning}

\subsubsection{Model Explainability using SHAP}

\subsection{Federated Learning}

\subsection{Attack Vectors in FL}





