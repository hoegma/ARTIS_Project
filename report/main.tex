% This is samplepaper.tex, a sample chapter demonstrating the
% LLNCS macro package for Springer Computer Science proceedings;
% Version 2.21 of 2022/01/12
%
\documentclass[runningheads]{llncs}
%
\usepackage[T1]{fontenc}
% T1 fonts will be used to generate the final print and online PDFs,
% so please use T1 fonts in your manuscript whenever possible.
% Other font encondings may result in incorrect characters.
%
\usepackage{graphicx}
\usepackage{hyperref}
\usepackage{enumitem}
\usepackage{fancyhdr}
\usepackage{titlesec}
\usepackage{parskip}
%\usepackage{enumerate}
\usepackage{enumitem}

\usepackage[backend=biber]{biblatex}
\addbibresource{references.bib}

\geometry{margin=1in}
\setlength{\parskip}{1em}
\setlength{\parindent}{0pt}

\pagestyle{fancy}
\fancyhf{}
\rhead{\textit{Advanced Research Topics in IT Security (ARTIS)}}
\lhead{\textit{Project Report}}
\rfoot{\thepage}

\titleformat{\section}{\large\bfseries}{\thesection.}{0.5em}{}
\titleformat{\subsection}{\normalsize\bfseries}{\thesubsection.}{0.5em}{}

\title{
    \vspace{-2cm}
    \textbf{Project Report – Privacy-Preserving and Explainable Federated Learning for Robust Digital Forensics}\\
    \large{Advanced Research Topics in IT Security (ARTIS)}\\
    \vspace{0.5cm}
}
\author{
    \begin{tabular}{lll}
        \textit{Omar Abushanab} & \textit{(omar.abushanab@student.guc.edu.eg)} \\
        \textit{Ibrahim Selim} & \textit{(mail)} \\
        \textit{Malak Abdelaziz} & \textit{(malak.abdelaziz@student.guc.edu.eg)} \\
        \textit{Peter Schropp} & \textit{(mail)} \\
        \textit{Matthias Högel} & \textit{(matthias.hoegel@uni-ulm.de)} \\
    \end{tabular}
}
\date{\today}

% Used for displaying a sample figure. If possible, figure files should
% be included in EPS format.
%
% If you use the hyperref package, please uncomment the following two lines
% to display URLs in blue roman font according to Springer's eBook style:
%\usepackage{color}
%\renewcommand\UrlFont{\color{blue}\rmfamily}
%
\begin{document}
%
\title{Privacy-Preserving and Explainable Federated Learning for Robust Digital Forensics}
%
%\titlerunning{Abbreviated paper title}
% If the paper title is too long for the running head, you can set
% an abbreviated paper title here
%
\author{Omar Abushanab\inst{1} \and
Matthias Högel\inst{2} \and
Malak Abdelaziz\inst{1} \and Ibrahim Selim\inst{1}}
%
%\authorrunning{F. Author et al.}
\authorrunning{}
% First names are abbreviated in the running head.
% If there are more than two authors, 'et al.' is used.
%
\institute{German University in Cairo, New Cairo City, Egypt \and
Ulm University, Albert-Einstein-Allee 5, 89081 Ulm, Germany
}
%
\maketitle              % typeset the header of the contribution
%
\begin{abstract}
The abstract should briefly summarize the contents of the paper in
150--250 words.

\keywords{First keyword  \and Second keyword \and Another keyword.}
\end{abstract}


\section{Introduction}
\label{sec:introduction}

Electronic devices are now involved in 85\% of all criminal investigations.
Each device generates and stores data, leaving behind a digital footprint~\cite{casino2022research}. 
This footprint can be used as evidence in subsequent investigations and consists of GPS coordinates, text messages, wearables and more~\cite{iotbds17}cite{casino2022research}.
In the age of big data, the amount of data is growing exponentially, making it much more difficult to search for important evidence and human capacity is not sufficient.
Machine learning (ML) is intended to remedy this lack of capacity and the increasing complexity of data.
ML models can be used to search for complex patterns and anomalies in large data sets~\cite{wang2019development}.
However, privacy and data security are also playing an increasingly important role.
In 2018, the GDPR was published in Europe to protect the data of individuals.
Data may not be shared between organizations without consent in order to train ML models.
This leads to data islands, which are a problem for classic ML models that require a large data set for the training process.
Federated learning aims to counteract these data islands by pursuing a decentralized approach to training an ML model.
Each client owns a portion of the data and trains an ML model locally.
After the training phase, all clients send their learned parameters to a server, which aggregates these parameters from all clients.
This method allows ML models to be trained in a decentralized manner without having to share client data.
However, this method also has vulnerabilities that can be exploited by attackers.
Attackers can attempt to disrupt the training process or draw conclusions about training data by reading the clients' parameters.~\cite{zhang2021survey}
\newline
The DF has a multimedia division that specializes in image forgery.
This also includes the detection of fake faces.
Training a fake face detection model requires real faces, which constitute sensitive data and cannot be shared without further ado.
That is why a decentralized approach using FL is one way to connect the data islands between organizations.~\cite{alashjaee2025machine}
\newline
The contributions of our project are as follows:
\begin{enumerate}
    \item We want to analyze existing forensic ML models in the field of fake face detection and reimplement one of them in an FL architecture without reducing accuracy.
    \item We want to develop a XAI module that provides insights into the model's decisions. This should reveal whether FL leads to different decisions being made.
    \item The FL model should be protected against attackers by integrating robust defense mechanisms.
\end{enumerate}

The rest of the project report is structured as follows.
Section 1 provides an overview of existing central ML models in the field of fake face detection.
The following section 2 introduces the basic concepts.
This is followed by a description of an ML model and how it is converted into FL, as well as the presentation of a simulation to simulate the training process.
Section 4 presents the results, which are then discussed and classified in section 5.
\section{Related Work}
\label{sec:related_work}

In the literature, there are two main approaches for a fake face detection without FL.
\newline
The first group of literature uses CNNs as feature extractors and trains ML models with these features for the classification task.
\newline
In~\cite{alashjaee2025machine}, the authors use the VGG16 model and initialize it with the weights trained on ImageNet.
The VGG16 consists of 13 convolutional layers, which extract the features of the input image.
The output of the 13th layer is flattened and consists of 73984 features, which represent the high-level features of the image.
The 140-real-fake-faces dataset is used as input and the features are extracted for each image.
These features are then used to train various ML models such as 
logistic regression,  K-means (KNN), decision trees, artificial neural networks (ANN) and random forests (RF).
RF performs best with an accuracy of 78.6\%, 76\% precision, 79\% recall and an F1 score of 77.4\%.
\newline
In~\cite{rafique2023deep}, the authors use the same approach, but the 
GoogLeNet, ResNet18 and SqueezeNet models are used to extract the features.
In addition, the authors use the 140k-real-fake-faces-with-ELA dataset, in which all images are preprocessed.
This preprocessing step is error level analysis (ELA), which is a forensic technique 
for examining image segments for varying compression levels 
that arise during digital editing of images.
The models again extract the high-level features of the images, which are used to train KNN and Support Vector Machine (SVM) models.
The combination of ResNet18 and SVM performs best in fake face detection
with an accuracy of 88.6\%, 88.5\% precision, 89\% recall and 85\% f1-score.
\newline
The second group of fake face detection approaches does not use CNNs as feature extractors, 
but trains them and uses them for classification.
Almost all of the papers mentioned use the 140k real-fake-faces dataset, which makes comparison easier.
\newline
In~\cite{jabbarli2024lightffdnets}, the authors developed a very lightweight fake face detection system that uses LightFDDNetv1 and v2,
which contain 3 and 5 convolutional layers and one output layer.
These models contain very few parameters so that they can be used on edge devices.
Both models were trained using transfer learning.
LightFDDNetv1 achieved a test accuracy of 69.9\%, 62\% precision, 85\% recall and an f1-score of 71.2\% after 10 epochs
and LightFDDNetv2 accuracy of 71.2\%, 78.2\% precision, 71\% recall and an f1-score of 74.5\%.
\newline
In~\cite{csafak2024detection}, the authors used the stacking ensemble learning method,
in which several models are trained separately from each other on the same dataset and the predictions are combined during the prediction process.
The models used were EfficientNetB0, MobileNetv1 and MobileNetv2, which were trained on the 
FFHQ dataset using transfer learning.
This enabled the authors to achieve an accuracy of 96.4\%, 97.8\% precision, 97.4\% recall and 97.6\% f1-score.
\newline
In~\cite{khudeyer2023fake}, the authors use EfficientB0 with transfer learning.
They use a learning rate scheduling technique to adjust the learning rate based on the training epoch.
This allows them to achieve an accuracy of 99.06\% and a loss of 0.057.
\section{Background}
\label{sec:Background}

\subsection{Digital Forensics}

\subsection{Federated Learning}

\subsection{Attack Vectors in FL}

\subsection{CNNs}

\subsection{Fundamentals}

\subsection{Transfer Learning}

\subsection{Model Explainability using SHAP}
\section{Methodology}
\label{sec:old-methods}

\subsection{Centralized Fake Face Detection Model}


The authors of~\cite{khudeyer2023fake} developed a method for fake face detection using CNN, which achieves an accuracy of 99.06%.
This approach is to be used as a benchmark for comparing central fake face detection with fake face detection using FL.
This work was reimplemented for verification purposes in order to ensure a meaningful comparison.
\newline
The dataset used was the well-known 140k real-fake faces dataset, which consists of 70,000 real faces and 70,000 fake faces.
The dataset was divided into 100.000 training images, 20.000 test images, and 20.000 validation images.
EfficientNetB0 with transfer learning was used as the model.
For this purpose, the model was initialized with the pretrained weights of EfficientNetB0 on the ImageNet dataset.
A lightweight head was attached to the pre-trained base model, consisting of global average pooling, a 256-dimensional fully connected layer with ReLU activation, batch normalization, and dropout, followed by a 2 dimensional softmax output layer.
The output is a 2-dimensional vector with probabilities indicating whether the input image is a fake or real face.
The model was optimized with binary cross-entropy loss and the Adam optimizer.
Training was performed with a batch size of 32 over 30 epochs with early stopping to reduce training time.
\newline
The paper presented a learning rate scheduler that adjusts the learning rate during training based on the epoch, as shown in Table~\ref{tab:learning_rate_scheduler}.
\begin{table}[t]
    \centering
    \begin{tabular}{l c}
    \hline
    \textbf{Epoch} & \textbf{Learning rate} \\
    \hline
    $epoch \leq 2$ & 0.01 \\
    $2 < epoch \leq 15$ & 0.001 \\
    $epoch > 15$ & 0.0001 \\
    \hline
    \end{tabular}
    \caption{Adjustment of the learning rate during training.}
    \label{tab:learning_rate_scheduler}
\end{table}
The learning rate scheduler ensures that significant weight adjustments are made early in training.
Furthermore, in later iterations, a strong adjustment is prevented by the decreasing learning rate.
This leads to faster convergence in early epochs, while weight optimizations can be performed in later itterations.
\newline
Due to time constraints, the model was only trained once, as training was very computationally intensive due to the large amount of data.

\subsection{Decentralized Fake Face Detection using FL}

In this section a decentralized trained fake face detection model is to be developed.
The training process will be explained below.
In addition, we will discuss how privacy attacks can be prevented during training.

\subsubsection{Research Scenario}

The following scenario is fictional.
\newline
Several research organizations want to work together to train a fake face detection model.
Each individual organization has images of fake faces, but also images of faces that belong to their customers.
An ML model should be trained together that can distinguish between real and fake images.
To do this, a large data set containing all images would have to be created in order to train the model.
However, all organizations are interested in protecting the privacy of their customers and therefore do not want to share the real images.
The solution is to train the model using FL.
This fake face detection model should be trained with all data from all organizations and should be available to everyone
without the need to share data between organizations.

\subsubsection{Thread Model}

All participants in the fake face detection training process are trusted.
This means that all model performance attacks that seek to undermine the convergence of the global model can be excluded.
After training, participants send their model weights to a trusted third-party server.
This ensures the secure aggregation of weights.
Weights sent by participants to the trusted third party could be captured during transmission.
Capturing the weights of individual organizations represents an attack vector for privacy attacks.
This attack vector should be reduced by the proposed encryption.

\subsubsection{FL attack mitigations}

\subsection{Explainability using SHAP}







% This are the results of the paper.
\section{Results and Discussion}
\label{sec:Results}

This section presents and discusses the experimental results obtained from evaluating both the centralized baseline model and the federated learning (FL) model with encrypted weights. We provide a quantitative comparison of key performance metrics, including accuracy, precision, recall, F1-score, and ROC-AUC. Additionally, we analyze model interpretability through SHAP explanations to understand how each model makes its predictions.

\subsection{Quantitative Results Comparison}
Table~\ref{tab:results_comparison} summarizes the evaluation results for the centralized baseline model and the federated learning model. The centralized model achieved an accuracy of 0.8460, an F1-score of 0.8460, and a ROC-AUC of 0.9276, demonstrating strong and balanced performance across both Real and Fake classes.

Notably, the federated model outperformed the centralized baseline across all major metrics, achieving an accuracy of 0.8558, precision of 0.8577, F1-score of 0.8556, and a ROC-AUC of 0.9358. This indicates that the encrypted federated training process not only preserves model utility but also leads to a modest improvement in generalization performance. This improvement can be attributed to the aggregation of diverse data distributions across clients, which enables the model to learn more robust and transferable feature representations.

The confusion matrix analysis further supports this observation. As shown in Figure~\ref{fig:confusion_matrix}, the federated model correctly classified 8,927 Fake samples compared to 8,236 for the centralized model, significantly reducing false acceptances of fake faces. Although the federated model produced slightly more false rejections of real faces (1,812 vs. 1,315), it substantially lowered the number of fake samples misclassified as real (1,073 vs. 1,764), which is particularly important in security-sensitive deepfake detection scenarios.

Overall, the federated model achieves better discrimination between Real and Fake classes, as reflected by its higher ROC-AUC and precision. These results demonstrate that federated learning with encrypted weight aggregation can match and even surpass centralized training, while providing strong privacy guarantees and avoiding direct data sharing.

\begin{table}[h]
    \centering
    \caption{Performance Comparison of Model Versions}
    \label{tab:results_comparison}
    \begin{tabular}{lccccc}
        \hline
        \textbf{Model Version} & \textbf{Accuracy} & \textbf{Precision} & \textbf{Recall} & \textbf{F1-Score} & \textbf{ROC-AUC} \\
        \hline
      Baseline (Centralized) & 0.8460 & 0.8467 & 0.8460 & 0.8460 & 0.9276 \\
FL with Weight Encryption & 0.8558 & 0.8577 & 0.8558 & 0.8556 & 0.9358 \\
        \hline
       
    \end{tabular}
\end{table}

\begin{figure}[H]
    \centering
    \includegraphics[width=0.9\textwidth]{figures/Confusion_Matrices.png}
    \caption{Confusion Matrices for FL Model and Baseline Model.}
    \label{fig:confusion_matrix}
\end{figure}



\subsection{SHAP Results and Model Interpretability}

To interpret and compare how the centralized and federated models make their predictions, we employ SHAP, which assigns an importance value to each pixel indicating how strongly it contributes to a model’s output. This allows us to analyze not only whether the two models produce similar predictions, but also whether they rely on the same visual evidence when classifying real and fake faces.

\subsubsection{Local SHAP Explanations}
The local SHAP maps for two representative samples—one Fake (Sample 65) and one Real (Sample 7)—are shown in the top rows of the figures \ref{fig:shap_sample65}, \ref{fig:shap_sample7}. These maps visualize which pixels most influenced each individual prediction. In both samples, the centralized and federated models focus on similar facial regions, particularly around the eyes, nose, cheeks, and mouth.

For Sample 65 shown in Figure \ref{fig:shap_sample65}, both models correctly classify the image as fake, although the centralized model is more confident (92.14\%) than the federated model (76.26\%). Despite this difference in confidence, their SHAP maps are highly aligned, indicating that both models are using the same underlying facial cues to detect manipulation. For Sample 7 shown in Figure \ref{fig:shap_sample7}, both models predict the image as real with nearly identical confidence 97.30\% vs. 97.03\%, and the SHAP maps are visually almost indistinguishable, further confirming consistent decision logic.

To further analyze explanation alignment, the top 5\% most influential pixels were extracted and compared between the two models. The resulting visualization categorizes regions that are important only to the centralized model, only to the federated model, and those identified as important by both. The dominant presence of overlapping regions demonstrates that both models consistently focus on the same key facial features, particularly around the eyes, cheeks, nose, and mouth—areas where deepfake generation methods commonly introduce subtle artifacts. Regions unique to either model are sparse and dispersed, suggesting only minor differences in attribution strength rather than fundamentally different decision strategies.

The SHAP difference maps are sparse and low in magnitude, showing that there are only minor variations in pixel-level importance between the two models.


\begin{figure}[H]
    \centering
    \includegraphics[width=0.8\textwidth]{figures/Sample_65.png}
   \caption{Local SHAP maps for Sample 65 (Fake).}
    \label{fig:shap_sample65}
\end{figure}

\begin{figure}[H]
    \centering
    \includegraphics[width=0.8\textwidth]{figures/Sample_7.png}
   \caption{Local SHAP maps for Sample 7 (Real).}
    \label{fig:shap_sample7}
\end{figure}



\subsubsection{Global SHAP Explanations}

Beyond individual sample analysis, global SHAP maps were computed to capture the average importance of each pixel across a large number of images, providing insight into what each model has learned to be generally important for classification. As shown in Figure \ref{fig:global_shap}, both centralized and federated models, the global SHAP maps strongly emphasize the central facial region, including the eyes, nose bridge, mouth, cheeks, and overall face outline.

These facial regions are well known to contain critical identity cues and common deepfake artifacts, such as texture inconsistencies, blending errors, and subtle geometric distortions introduced during face synthesis. The strong visual similarity between the centralized and federated global SHAP maps indicates that federated training preserves the same high-level feature representations learned through centralized training. This observation is further reinforced by the global SHAP difference maps, which exhibit only minor, spatially scattered variations and no systematic shift in attention or semantic focus.
%
This strong spatial agreement is also supported quantitatively by a SHAP cosine similarity of 0.9961, indicating near-perfect alignment between the pixel-level explanations of the two models. While the centralized model exhibits slightly higher mean absolute SHAP values, reflecting marginally stronger feature weighting, the overall attribution patterns remain highly consistent. Together, these findings confirm that federated learning preserves global explanation structure and semantic focus comparable to centralized training.

\begin{figure}[H]
    \centering
    \includegraphics[width=0.8\textwidth]{figures/Global_SHAP.png}
    \caption{Global SHAP attribution maps for centralized and federated models.}
    \label{fig:global_shap}
\end{figure}

\section{Discussion}
\label{sec:Discussion}

This is the discussion of the paper.
\section{Conclusion}
\label{sec:Conclusion}

This is the conclusion of the paper.



%
% ---- Bibliography ----
%
% BibTeX users should specify bibliography style 'splncs04'.
% References will then be sorted and formatted in the correct style.
%
\bibliographystyle{splncs04}
\bibliography{references}

\end{document}
